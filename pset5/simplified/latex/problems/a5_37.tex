\section*{A5.37 Properties under Slater's condition}

Consider a convex optimization problem
\[
\begin{aligned}
\text{minimize} \quad & f_0(x) \\
\text{subject to} \quad & f_i(x) \leq 0, \quad i = 1, \ldots, m \\
& Ax = b,
\end{aligned}
\]
with variable $x \in \mathbf{R}^n$, that satisfies Slater's constraint qualification. Determine whether each of the statements below is true or false. True means it holds with no further assumptions.

\textbf{(a) The primal and dual problems have the same objective value.}

\textbf{True.}

Slater's constraint qualification states that there exists a point $\tilde{x}$ in the relative interior of the domain such that:
\begin{itemize}
    \item $f_i(\tilde{x}) < 0$ for all $i = 1, \ldots, m$ (strict inequality for inequality constraints)
    \item $A\tilde{x} = b$
\end{itemize}

For convex optimization problems, \textbf{Slater's condition implies strong duality}. Therefore:
\[
p^\star = d^\star
\]

The primal and dual optimal values are equal.

\textbf{(b) The primal problem has a unique solution.}

\textbf{False.}

Slater's condition guarantees strong duality, but it does \emph{not} guarantee uniqueness of the primal solution.

\textbf{Counterexample:} Consider the LP
\[
\begin{aligned}
\text{minimize} \quad & x_1 \\
\text{subject to} \quad & x_1 + x_2 \leq 2 \\
& x_1 \geq 0, \; x_2 \geq 0
\end{aligned}
\]

This is convex and satisfies Slater's condition (e.g., $\tilde{x} = (0.5, 0.5)$ is strictly feasible). However, the optimal value is $p^\star = 0$, achieved by \emph{any} point of the form $(0, t)$ where $0 \leq t \leq 2$. The solution is not unique.

\textbf{(c) The dual problem is not unbounded.}

\textbf{True.}

For a convex optimization problem satisfying Slater's condition:
\begin{itemize}
    \item The primal problem is feasible (Slater's condition provides a feasible point).
    \item By weak duality, $d^\star \leq p^\star$.
\end{itemize}

The dual problem is a \emph{maximization} problem:
\[
\text{maximize} \quad g(\lambda, \nu) \quad \text{subject to} \quad \lambda \succeq 0
\]

Since $g(\lambda, \nu) \leq p^\star$ for all dual feasible $(\lambda, \nu)$, the dual objective is \textbf{bounded above} by $p^\star$.

Therefore, the dual problem cannot be unbounded (i.e., $d^\star \neq +\infty$).

\textbf{(d) Suppose $x^\star$ is optimal, with $f_1(x^\star) = -0.2$. Then for every dual optimal point $(\lambda^\star, \nu^\star)$, $\lambda_1^\star = 0$.}

\textbf{True.}

This follows from \textbf{complementary slackness}. Under Slater's condition, strong duality holds, and the KKT conditions are necessary and sufficient for optimality. One of the KKT conditions is:
\[
\lambda_i^\star f_i(x^\star) = 0, \quad i = 1, \ldots, m
\]

For $i = 1$: Since $f_1(x^\star) = -0.2 \neq 0$, it follows that:
\[
\lambda_1^\star \cdot (-0.2) = 0 \quad \Longrightarrow \quad \lambda_1^\star = 0
\]

Intuitively, $f_1(x^\star) = -0.2 < 0$ means the first inequality constraint is \textbf{inactive} (strictly satisfied) at the optimal point. By complementary slackness, the corresponding dual variable must be zero.

This holds for \emph{every} dual optimal point $(\lambda^\star, \nu^\star)$.
