\section*{A17.30 Maximum Sharpe ratio portfolio}

%Consider a portfolio with weights $x \in \mathbf{R}^n$, mean return $\mu \in \mathbf{R}^n$, and covariance $\Sigma \in \mathbf{S}^n_{++}$.

%The Sharpe ratio is:
%\[
%S(x) = \frac{\mu^T x}{\|\Sigma^{1/2} x\|_2}
%\]

\%textbf{Problem:} Maximize $S(x)$ subject to $\mathbf{1}^T x = 1$ and $\|x\|_1 \leq L^{\max}$.

%\subsection*{(a) Quasiconvexity}
\subsection*{(a)}

The Sharpe ratio $S(x)$ is quasiconvex for $\mu^T x > 0$.

The sublevel set $\{x : S(x) \leq t\}$ for $t > 0$ is:
\[
\{x : \mu^T x \leq t \|\Sigma^{1/2} x\|_2\}
\]

This is equivalent to $\|\Sigma^{1/2} x\|_2 \geq (\mu^T x)/t$, which defines a convex set (second-order cone constraint). Hence $S(x)$ is quasiconvex.

\subsection*{(b) Transformation to convex problem}

Let $y = x / (\mu^T x)$ (for $\mu^T x > 0$). Then:
\begin{itemize}
    \item $\mu^T y = 1$
    \item $\mathbf{1}^T y = 1/(\mu^T x)$
    \item $S(x) = 1/\|\Sigma^{1/2} y\|_2$
\end{itemize}

Maximizing $S(x)$ is equivalent to minimizing $\|\Sigma^{1/2} y\|_2$ subject to $\mu^T y = 1$ and the transformed leverage constraint $\|y\|_1 \leq L^{\max} \cdot \mathbf{1}^T y$.

\textbf{Results} (for $n=10$ assets, $L^{\max} = 1.5$):
\begin{itemize}
    \item Expected return: 0.129
    \item Standard deviation: 0.061
    \item Sharpe ratio: $\boxed{2.13}$
    \item Leverage $\|x\|_1$: 1.32
\end{itemize}

\textbf{Code:}
\begin{lstlisting}[language=Python, basicstyle=\ttfamily\footnotesize]
y = cp.Variable(n)
constraints = [mu @ y == 1, cp.sum(y) >= 0,
               cp.norm1(y) <= L_max * cp.sum(y)]
prob = cp.Problem(cp.Minimize(cp.quad_form(y, Sigma)), constraints)
prob.solve()
x_opt = y.value / np.sum(y.value)
\end{lstlisting}