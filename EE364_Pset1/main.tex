\documentclass[11pt]{article}
\usepackage{amsmath,amssymb,amsthm}
\usepackage{geometry}
\usepackage{mathtools}
\usepackage{enumitem}
\usepackage{fancyhdr}
\usepackage{lastpage}
\usepackage{xcolor}
\usepackage{subcaption}
\geometry{margin=1in}

\pagestyle{fancy}
\fancyhf{} % clear default header/footer

\fancyhead[L]{EE 364A}
\fancyhead[C]{Victor Sebastian Martinez Perez}
\fancyhead[R]{Problem Set 1}

\renewcommand{\headrulewidth}{0.4pt} % underline thickness
\renewcommand{\footrulewidth}{0pt}   % no footer line

\begin{document}

\section*{2.7 Voronoi description of a halfspace}

\begin{align*}
\|x-a\|_2^2 &\leq \|x-b\|_2^2 \\
(x-a)^T(x-a) &\leq (x-b)^T(x-b) \\
x^Tx - 2a^Tx + a^Ta &\leq x^Tx - 2b^Tx + b^Tb \\
2(b-a)^T x &\leq b^Tb - a^Ta
\end{align*}

Thus the set can be written as the halfspace, shown in Fig. \ref{fig:2.7}
\[
\{x \mid c^T x \le d\}
\quad \text{with} \quad
c = 2(b-a), \quad d = b^Tb - a^Ta.
\]

\begin{figure} [h]
    \centering
    \includegraphics[width=0.4\linewidth]{imagen.png}
    \caption{Halfspace describing $c^T x \le d$}
    \label{fig:2.7}
\end{figure}

\section*{2.12 Which of the following sets are convex?}

\subsection*{(a) Slab \hfill \textbf{\textcolor{red}{(Convex)}}}
\[
S = \{x \in \mathbb{R}^n \mid \alpha \le a^T x \le \beta\}
\]

\subsubsection*{Line-segment}
Let $x_1, x_2 \in S$ and $\theta \in [0,1]$. Define
\[
x_\theta = \theta x_1 + (1-\theta)x_2.
\]
Then
\[
a^T x_\theta = \theta a^T x_1 + (1-\theta)a^T x_2.
\]
Since $\alpha \le a^T x_1 \le \beta$ and $\alpha \le a^T x_2 \le \beta$,
\[
\alpha \le a^T x_\theta \le \beta,
\]
so $x_\theta \in S$. Hence $S$ is convex.

\subsubsection*{Intersection of halfspaces}
Alternatively,
\[
S = \{x \mid a^T x \le \beta\} \cap \{x \mid -a^T x \le -\alpha\},
\]
which is an intersection of two halfspaces, hence convex.

\subsection*{(b) Rectangle \hfill \textbf{\textcolor{red}{(Convex)}}}
\[
R = \{x \in \mathbb{R}^n \mid \alpha_i \le x_i \le \beta_i,\ i=1,\dots,n\}
\]

\subsubsection*{Line-segment}
Let $x_1, x_2 \in R$ and $\theta \in [0,1]$. Define
\[
x_\theta = \theta x_1 + (1-\theta)x_2.
\]
For each coordinate $i$,
\[
x_{\theta,i} = \theta x_{1,i} + (1-\theta)x_{2,i}.
\]
Since $\alpha_i \le x_{1,i} \le \beta_i$ and $\alpha_i \le x_{2,i} \le \beta_i$,
\[
\alpha_i \le x_{\theta,i} \le \beta_i,
\]
So each $x_{\theta,i} \in R$. Hence $R$ is convex.

\subsection*{(c) Wedge \hfill \textbf{\textcolor{red}{(Convex)}}}
\[
W = \{x \in \mathbb{R}^n \mid a_1^T x \le b_1,\ a_2^T x \le b_2\}
\]

\subsubsection*{Intersection of halfspaces}
Each inequality defines a halfspace. Since
\[
W = \{x \mid a_1^T x \le b_1\} \cap \{x \mid a_2^T x \le b_2\},
\]
$W$ is convex.

\subsection*{(d) Points closer to a fixed point than to a set \hfill \textbf{\textcolor{red}{(Convex)}}}
\[
Z = \{x \mid \|x - x_0\|_2 \le \|x - y\|_2 \ \forall y \in S\}.
\]
For a fixed $y \in S$,
\begin{align*}
\|x - x_0\|_2^2 &\le \|x - y\|_2^2 \\
(x-x_0)^T(x-x_0) &\le (x-y)^T(x-y) \\
-2x_0^Tx + x_0^Tx_0 &\le -2y^Tx + y^T y \\
2(y-x_0)^T x &\le y^T y - x_0^T x_0.
\end{align*}
Each inequality defines a halfspace. Since
\[
Z = \bigcap_{y \in S} \{x \mid 2(y-x_0)^T x \le y^T y - x_0^T x_0\},
\]
$Z$ is an intersection of halfspaces, hence convex.

\section*{2.15 Sets of probability distributions}

\subsection*{(a) $\alpha \le \mathbb{E}[f(x)] \le \beta$ \hfill \textbf{\textcolor{red}{(Convex)}}}

\[
\mathbb{E}[f(x)] = \sum_{i=1}^n p_i f(a_i) = c^T p,
\]
where $c = [f(a_1), \dots, f(a_n)]^T$. Thus,
\[
\alpha \le c^T p \le \beta
\]
defines a slab, i.e., the intersection of two halfspaces, which is convex.

The feasible set
\[
\{p \mid \alpha \le \mathbb{E}[f(x)] \le \beta,\ \mathbf{1}^T p = 1,\ p \succeq 0\}
\]
is the intersection of this slab, the probability simplex hyperplane, and the nonnegative orthant. Since each set is convex, their intersection is convex.

\subsection*{(b) $\textbf{prob}(x > a) \le \beta$ \hfill \textbf{\textcolor{red}{(Convex)}}}

\[
\textbf{prob}(x > a) = \sum_{i : a_i \ge a} p_i = c^T p,
\]

where $c \in \mathbb{R}^n$ is defined componentwise by
\[
c_i =
\begin{cases}
1, & a_i \ge a,\\
0, & a_i < a.
\end{cases}
\]

So, the constraint $\textbf{prob}(x > a) \le \beta$ defines a halfspace. And the intersection of the halfspace with the probability simplex $P$ is also convex.

\subsection*{(c) $\mathbb{E}|x^3| \le \alpha \mathbb{E}|x|$ \hfill \textbf{\textcolor{red}{(Convex)}}}

\begin{align*}
    \mathbb{E}|x^3| - \alpha \mathbb{E}|x| &\le 0 \\
    \sum_i (|a_i^3| p_i - \alpha |a_i|p_i) &\le 0 \\
    \sum_i (|a_i^3| - \alpha |a_i|) p_i &\le 0 \\
    c^Tp &\le 0 
\end{align*}

where the elements of $c$ are $c_i = |a_i|^3 - \alpha|a_i|$, which is a halfspace in p. Intersecting with the probability simplex preserves convexity.

\subsection*{(d) $\mathbb{E}[x^2] \le \alpha$ \hfill \textbf{\textcolor{red}{(Convex)}}}

\[
\mathbb{E}[x^2] = \sum_{i=1}^n a_i^2 p_i = c^T p,
\]
where $c = [a_1^2,\dots,a_n^2]^T$. Thus the constraint is $c^T p \le \alpha$, defining a halfspace in $p$. Intersecting with the probability simplex preserves convexity.

\subsection*{(e) $\mathbb{E}[x^2] \ge \alpha$ \hfill \textbf{\textcolor{red}{(Convex)}}}

\[
\mathbb{E}[x^2] = \sum_{i=1}^n a_i^2 p_i = c^T p,
\]

so the constraint is $-c^T p \le -\alpha$, which is also a halfspace (the other side of the hyperplane). Intersecting with the probability simplex preserves convexity.


\subsection*{(f) $\mathrm{Var}(x) \le \alpha$ \hfill \textbf{\textcolor{red}{(Non-convex)}}}

\[
\mathrm{Var}(x) = \mathbb{E}[x^2] - (\mathbb{E}[x])^2
= s^T p - (a^T p)^2.
\]

where $s=[a_1^2,\dots,a_n^2]^T$ and $a=[a_1,\dots,a_n]^T$. So, the constraint becomes $s^T p - (a^T p)^2 \le \alpha$, composed of a linear minus a convex term, hence concave.

Example: Let $x \in \{-1,1\}$.
\begin{align*}
    p_1 &= (1,0), \quad \text{var}_1=1-(-1)^2=0 \qquad (\text{always} \quad x=-1) \\
    p_2 &= (0,1), \quad \text{var}_2=1-(1)^2=0 \qquad (\text{always} \quad x=1)
\end{align*}

Their midpoint
\[
p_3 = (1/2,1/2)
\]
has variance $1$, violating convexity.

\subsection*{(g) $\mathrm{Var}(x) \ge \alpha$ \hfill \textbf{\textcolor{red}{(Convex)}}}

\begin{align*}
\mathrm{Var}(x) &= \mathbb{E}[x^2] - (\mathbb{E}[x])^2 = s^T p - (a^T p)^2 \\
&= s^T p - (a^T p)(a^T p) = s^T p - p^T (aa^T) p \\
&= s^T p - p^T Ap \ge \alpha
\end{align*}

Let $A = aa^T \succeq 0$. Since $(a^T p)^2 = p^T A p$, the constraint becomes
\[
p^T A p - s^T p + \alpha \le 0
\]

Since $A \succeq 0$, $p^T A p - s^T p + \alpha \le 0$ is convex, and the set, and the intersection with the probability simplex preserves convexity.

\section*{A1.1 Convex Optimization}

\begin{itemize}
\item (a) True
\item (b) True
\item (c) True
\item (d) True
\end{itemize}

\section*{A1.2 Device Sizing}

\begin{itemize}
\item (a) False
\item (b) True
\item (c) True
\end{itemize}

\section*{A1.4 Local Optimization}

\begin{itemize}
\item (a) True
\item (b) False
\item (c) False
\end{itemize}

\section*{A2.5 Convex and conic hull}

\begin{itemize}
\item (a) True
\[
(0,-1/3)= \frac{1}{3}(1,0)+0(1,1)+\frac{1}{3}(-1,-1)+\frac{1}{3}(0,0)
\]
and $\frac{1}{3}+\frac{1}{3}+\frac{1}{3}+0=1$ and non-negative.
\item (b) False
\[
(0,1/3)= -\frac{1}{3}(1,0)+0(1,1)-\frac{1}{3}(-1,-1)+\frac{1}{3}(0,0)
\]
but coefficients are negative.
So, $(0,1/3)$ requires negative weights, hence it is not in the convex hull of C.
\item (c) False. (0,1/3) is not in the conic hull of C
\begin{figure} [h]
    \centering
    \includegraphics[width=0.25\linewidth]{conic_hull.png}
    \caption{Conic hull}
    \label{fig:conic_hull}
\end{figure}
\end{itemize}

\section*{A2.8 Convexity of some sets}

\begin{itemize}
\item (a) Convex. Fig. \ref{fig:a}
\item (b) Convex. Fig. \ref{fig:b}
\item (c) Nonconvex. Fig. \ref{fig:c}
\item (d) Convex. Fig. \ref{fig:d}
\end{itemize}

\begin{figure*}[t!]
    \centering
    \begin{subfigure}[t]{0.5\textwidth}
        \centering
        \includegraphics[width=0.5\linewidth]{A2_8a.jpg}
        \caption{$\{(x, y) \in \mathbb{R}^2_{++} \mid x/y \le 1\}$}
        \label{fig:a}
    \end{subfigure}%
    ~
    \begin{subfigure}[t]{0.5\textwidth}
        \centering
        \includegraphics[width=0.5\linewidth]{A2_8b.jpg}
        \caption{$\{(x, y) \in \mathbb{R}^2_{++} \mid x/y \ge 1\}$}
        \label{fig:b}
    \end{subfigure}

    \begin{subfigure}[t]{0.5\textwidth}
        \centering
        \includegraphics[width=0.5\linewidth]{A2_8c.jpg}
        \caption{$\{(x, y) \in \mathbb{R}^2_{+} \mid xy \le 1\}$}
        \label{fig:c}
    \end{subfigure}%
    ~
    \begin{subfigure}[t]{0.5\textwidth}
        \centering
        \includegraphics[width=0.5\linewidth]{A2_8d.jpg}
        \caption{$\{(x, y) \in \mathbb{R}^2_{+} \mid xy \ge 1\}$}
        \label{fig:d}
    \end{subfigure}
    
    \caption{Sets}
    \label{fig:a2_8_sets}
\end{figure*}

\section*{A2.9 Square and disk}
\begin{figure} [h]
    \centering
    \includegraphics[width=0.25\linewidth]{a29.png}
    \label{fig:a29}
\end{figure}

\begin{itemize}
\item (a) $S \cap D$ is convex.
\item (b) $S \cup D$ is convex.
\item (c) $S \setminus D$ is not convex.
\end{itemize}

\end{document}
