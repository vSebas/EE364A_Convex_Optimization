\section*{A4.17 Heuristic suboptimal solution for Boolean LP}

The LP relaxation gives lower bound $L = -34.42$. Threshold rounding with $\hat{x}_i = \mathbf{1}(x^{\text{rlx}}_i \ge t)$ yields feasible Boolean solutions for $t \ge 0.556$. The best upper bound $U = -33.58$ occurs at $t^* = 0.556$, with optimality gap $U - L = 0.84$ (only $2.4\%$ of $|L|$). As $t$ increases, fewer components are set to 1, reducing constraint violations but worsening the objective (since $c < 0$). The small gap indicates the heuristic finds a near-optimal solution.

\begin{figure}[H]
    \centering
    \includegraphics[width=0.85\textwidth]{img/a4_17.png}
    \caption{Objective value and max constraint violation}
\end{figure}

\subsection*{Code}
\begin{verbatim}
# LP relaxation
x = cp.Variable(n)
prob = cp.Problem(cp.Minimize(c @ x), [A @ x <= b, 0 <= x, x <= 1])
prob.solve()
x_rlx, L = x.value, prob.value

# Threshold rounding
t_vals = np.linspace(0, 1, 100)
objs = np.array([c @ (x_rlx >= t) for t in t_vals])
viols = np.array([np.max(A @ (x_rlx >= t) - b) for t in t_vals])
feas = viols <= 1e-6

# Best feasible
best_idx = np.where(feas)[0][np.argmin(objs[feas])]
U, t_best = objs[best_idx], t_vals[best_idx]
print(f"L = {L:.2f}, U = {U:.2f}, gap = {U-L:.2f}, t* = {t_best:.2f}")
\end{verbatim}

