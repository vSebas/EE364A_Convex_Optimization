\section*{A17.5 Simple portfolio optimization}

\textbf{(a) Minimum-risk portfolios} with same expected return as uniform portfolio ($= 0.0729$):%($\bar{p}^T x_{\text{unif}} = 0.0729$):

\begin{center}
\begin{tabular}{lcc}
\hline
Portfolio & Std Dev & Risk Reduction \\
\hline
Uniform $(x = \mathbf{1}/n)$ & 0.0666 & -- \\
No constraints & 0.0090 & 86.4\% \\
Long-only $(x \geq 0)$ & 0.0395 & 40.7\% \\
Short limit $(\mathbf{1}^T x_- \leq 0.5)$ & 0.0150 & 77.5\% \\
\hline
\end{tabular}
\end{center}

The unconstrained portfolio achieves the lowest risk by taking short positions. The long-only constraint is most restrictive. Allowing limited shorting (0.5) significantly reduces risk compared to long-only.

\textbf{(b) Efficient frontier:}

\begin{figure}[H]
    \centering
    \includegraphics[width=0.8\textwidth]{img/a17_5_frontier.png}
    \caption{Risk-return trade-off curves for long-only and short-limited portfolios.}
\end{figure}

The short-limited portfolio dominates the long-only portfolio (lower risk for same return). Both curves pass through lower-risk regions than the uniform portfolio.

\subsection*{Code}
\begin{verbatim}
# (a) Minimum-variance portfolio with target return
def min_var(r_target, long_only=False, short_limit=None):
    x = cp.Variable(n)
    cons = [cp.sum(x) == 1, pbar.flatten() @ x == r_target]
    if long_only: cons.append(x >= 0)
    if short_limit: cons.append(cp.sum(cp.neg(x)) <= short_limit)
    cp.Problem(cp.Minimize(cp.quad_form(x, S)), cons).solve()
    return x.value, np.sqrt(x.value @ S @ x.value)

_, std_long = min_var(r_unif, long_only=True)
_, std_short = min_var(r_unif, short_limit=0.5)

# (b) Efficient frontier
r_min, r_max = min(pbar), max(pbar)  # return range for long-only
returns = np.linspace(r_min, r_max, 50)
stds = [min_var(r, long_only=True)[1] for r in returns]
plt.plot(stds, returns)
\end{verbatim}

